\documentclass[12pt,a4paper,openright]{article}
\usepackage{gensymb}
\usepackage{amsmath}
\usepackage{amssymb}
\usepackage{enumitem}
\usepackage{graphicx}
\usepackage{sansmath}
\usepackage{pst-eucl}
\usepackage{float}
\usepackage[numbered,framed]{matlab-prettifier}
\usepackage[T1]{fontenc}
\usepackage{setspace}
\usepackage{sectsty}
\usepackage[colorlinks=true,linkcolor=blue,urlcolor=black,bookmarksopen=true]{hyperref}
\usepackage{bookmark}
\newcommand{\diff}[3]{\dfrac{d^{#3}{#1}}{d{#2}^{#3}}}
\newcommand{\question}[0]{\textbf{Question}: }
\newcommand{\itq}[0]{\item \question}
\newcommand{\answer}[0]{\textbf{Answer}: }
\newcommand{\letlist}{\begin{enumerate}[label=(\alph*)]}
\setlength{\parindent}{0pt}
\renewcommand{\baselinestretch}{1.5}

\begin{document}
	\[
	- \dfrac{d^2 y}{dx^2} + xy = \lambda y, \hspace{0.1cm}\mathrm{on} \hspace{0.1cm} x\in[0,\infty].
	\]
	
	With the boundary conditions $y(0)=y(\infty)=0$. And where $\lambda$ is the eigenvalue of this Sturm-Louiville problem. 
	
	The analytical solutions to this problem are:
	
	\[
	y_n(x) = \mathrm{Ai}(x-\lambda_n)
	\]
	
	where the eigenvalues, $\lambda_n$, are the negative of the zeros of the Airy Ai function.
\end{document}